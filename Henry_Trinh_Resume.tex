%!TEX TS-program = xelatex
%!TEX encoding = UTF-8 Unicode
% Awesome CV LaTeX Template for CV/Resume
%
% This template has been downloaded from:
% https://github.com/posquit0/Awesome-CV
%
% Original author:
% Claud D. Park <posquit0.bj@gmail.com>
% http://www.posquit0.com
%
% Modifications by:
% Henry Trinh <henrymtrinh14@gmail.com>
%
% Template license:
% CC BY-SA 4.0 (https://creativecommons.org/licenses/by-sa/4.0/)
%


%-------------------------------------------------------------------------------
% CONFIGURATIONS
%-------------------------------------------------------------------------------
% A4 paper size by default, use 'letterpaper' for US letter
\documentclass[12pt, letterpaper]{awesome-cv}

% Configure page margins with geometry
\geometry{left=0.5in, top=0.5in, right=0.5in, bottom=0.5in}

% Specify the location of the included fonts
\fontdir[fonts/]

% Color for highlights
% Awesome Colors: awesome-emerald, awesome-skyblue, awesome-red, awesome-pink, awesome-orange
%                 awesome-nephritis, awesome-concrete, awesome-darknight
\colorlet{awesome}{awesome-skyblue}
%\colorlet{awesome}{awesome-darknight}
% Uncomment if you would like to specify your own color
% \definecolor{awesome}{HTML}{CA63A8}

% Colors for text
% Uncomment if you would like to specify your own color
% \definecolor{darktext}{HTML}{414141}
% \definecolor{text}{HTML}{333333}
% \definecolor{graytext}{HTML}{5D5D5D}
% \definecolor{lighttext}{HTML}{999999}

% Set false if you don't want to highlight section with awesome color
\setbool{acvSectionColorHighlight}{true}

% If you would like to change the social information separator from a pipe (|) to something else
\renewcommand{\acvHeaderSocialSep}{\quad\textbar\quad}

\makeatletter
\patchcmd{\@sectioncolor}{\color}{\mdseries\color}{}{}
\makeatother

%-------------------------------------------------------------------------------
%	PERSONAL INFORMATION
%	Comment any of the lines below if they are not required
%-------------------------------------------------------------------------------
% Available options: circle|rectangle,edge/noedge,left/right
% \photo[rectangle,edge,right]{profile}
\name{Henry}{Trinh}
% \position{Software Engineer}
% \address{address}

\mobile{(408) 324-6330}
\email{henrymtrinh14@gmail.com}
\homepage{henrytrinh.xyz}
\github{thenry3}
\linkedin{thenry3}
% \gitlab{gitlab-id}
% \stackoverflow{SO-id}{SO-name}
% \twitter{@twit}
% \skype{skype-id}
% \reddit{reddit-id}
% \extrainfo{extra informations}

%-------------------------------------------------------------------------------
\begin{document}
% \vspace*{-0.35in}
% Print the header with above personal informations
% Give optional argument to change alignment(C: center, L: left, R: right)
\makecvheader
% \vspace*{-0.1in}
% Print the footer with 3 arguments(<left>, <center>, <right>)
% Leave any of these blank if they are not needed
% \makecvfooter
%   {\today}
%   {Henry Trinh~~~·~~~Résumé}
%   {\thepage}

%-------------------------------------------------------------------------------
%	CV/RESUME CONTENT
%	Each section is imported separately, open each file in turn to modify content
%-------------------------------------------------------------------------------

% EDUCATION
\cvsection{Education}
\begin{cventries}
  \cventry
    {Bachelor of Science in Computer Science} % Degree
    {University of California, Los Angeles} % Institution
    {Expected Graduation in June 2022} % Location
    {Major GPA: 3.9} % Date(s)
    {
      \vspace{-0.11in}
      \begin{cvskills}
        \cvskill
          {Courses} % Type
          {Data Structures, Algorithms, Operating Systems, Linear Algebra, Discrete Math, Software Construction Laboratory} % Skillset
      \end{cvskills}
    }
  \vspace{-0.2in}
\end{cventries}

% EXPERIENCE
\cvsection{Experience}
\begin{cventries}
  \cventry
    {Software Development Engineer Intern} % Job title
    {Amazon} % Organization
    {Seattle, WA} % Location
    {June 2020 - PRESENT} % Date(s)
    {
      \begin{cvitems} % Description(s) of tasks/responsibilities
        \item {Contributing in AWS Cryptography on the ACM PrivateCA team in Summer 2020}
      \end{cvitems}
    }

  \cventry
    {Research Assistant} % Job title
    {Center for Vision, Cognition, Learning, and Autonomy} % Organization
    {Los Angeles, CA} % Location
    {April 2020 - PRESENT} % Date(s)
    {
      \begin{cvitems} % Description(s) of tasks/responsibilities
        \item {Researching deep learning under Professor Song-Chun Zhu for path planning and trajectory prediction in self-driving vehicles}
        \item {Implemented LSTM neural network with social pooling to determine possible trajectories of human movement in dense crowds}
        \item {Improved performance of data feed pipelines for neural networks by creating a tool to preprocess raw data into loadable binary files, preventing redudant calculations}
      \end{cvitems}
    }

  \cventry
    {Software Engineer Intern} % Job title
    {Archanics} % Organization
    {El Segundo, CA} % Location
    {June 2019 - August 2019} % Date(s)
    {
      \begin{cvitems} % Description(s) of tasks/responsibilities
        \item {Developed Android GIS/GPS map application with real-time traffic analysis and 3-D geographical layers}
        \item {Programmed mapping functionalities to generate routes, invite people to events, and display multiple datasets}
      \end{cvitems}
    }

  \cventry
    {Lead Software Developer} % Job title
    {Daily Bruin} % Organization
    {Los Angeles, CA} % Location
    {April 2019 - June 2020} % Date(s)
    {
      \begin{cvitems} % Description(s) of tasks/responsibilities
        \item {Built database infrastructure and optimized database querying in API endpoints for multiple projects}
		    \item {Constructed customizable components for the Daily Bruin Lux library, resulting in faster website development}
		    \item {Increased number of average users by 10\% by improving mobile user experience in new interactive flat pages}
      \end{cvitems}
    }

\end{cventries}

% PROJECTS
\cvsection{Projects}
\begin{cventries}
  \cventry
    {} % Empty position
    {Simultaneous Location and Mapping (SLAM) Tool} % Project
    {} % Empty location
    {} % Empty date
    {
      \vspace{-0.2in}
      \begin{cvitems} % Description(s) bullet points
        \item {Developed tool to generate a three-dimensional map of an environment by analyzing any given video}
		    \item {Implemented an algorithm to extract features and camera pose from images to render points in 3-D space}
      \end{cvitems}
    }

  \cventry
    {} % Empty position
    {Tongits} % Project
    {} % Empty location
    {} % Empty date
    {
      \vspace{-0.2in}
      \begin{cvitems} % Description(s) bullet points
      	\item {Implemented Android application of a Filipino card game with AI opponents, music, and point scoring}
		    \item 
            Gained over \href{https://play.google.com/store/apps/details?id=com.creativelabs.tongits&hl=en_US/}{\textbf{12,000 downloads on Google Play Store}}
      \end{cvitems}
    }

  \cventry
    {} % Empty position
    {Drug Decider} % Project
    {} % Empty location
    {} % Empty date
    {
      \vspace{-0.2in}
      \begin{cvitems} % Description(s) bullet points
      	\item {Designed database schema and created REST API endpoints to connect the user interface with the machine learning model and database}
		    \item {Fixed bugs in machine learning model for predicting a patient’s treatment response to anti-psychotics}
        \item
            Used in production by the UCLA David Geffen School of Medicine at \href{https://drugdecider.org}{\textbf{drugdecider.org}}
      \end{cvitems}
    }

  % \cventry
  %   {}
  %   {Pinstagrad}
  %   {}
  %   {}
  %   {
  %     \vspace{-0.2in}
  %     \begin{cvitems} % Description(s) bullet points
  %     	\item {Built data pipeline to connect the database with cloud photo storage to maximize retrieval efficiency for displaying multiple photos for a graduation photo sharing platform}
	% 	    \item {Implemented user photo submission form with photo preview and tagging features}
  %     \end{cvitems}
  %   }
\end{cventries}

% SKILLS
\cvsection{Skills}
\begin{cvskills}
  \cvskill
    {Languages} % Type
    {Python, Java, C++, C, JavaScript, TypeScript, Go, HTML, CSS} % Skillset

  \cvskill
    {Frameworks + Tools} % Type
    {PyTorch, React, Django, Node, Git, Docker, OpenCV, MongoDB, mySQL, PostgreSQL, Numpy} % Skillset
  
  \cvskill
    {Technologies} % Type
    {Machine Learning, Deep Learning, Android, Computer Vision, SLAM, SQL, noSQL, Web Development} % Skillset
    
\end{cvskills}

%-------------------------------------------------------------------------------
\end{document}
